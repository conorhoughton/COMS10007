%3_big_oh.tex
%notes for the course COMS10007 taught at the University of Bristol
%2017/18 Conor Houghton conor.houghton@bristol.ac.uk

%To the extent possible under law, the author has dedicated all copyright 
%and related and neighboring rights to these notes to the public domain 
%worldwide. These notes are distributed without any warranty. 

\documentclass[11pt,a4paper]{scrartcl}
\typearea{12}
\usepackage{graphicx}
%\usepackage{pstricks}
\usepackage{listings}
\usepackage{color}
\lstset{language=C}
\pagestyle{headings}
\markright{COMS10007 - algorithms 3\_big\_oh (c) - Conor\footnote{\texttt{github.com/conorhoughton/COMS10007}}}
\begin{document}

\subsection*{3 Big Oh notation}

To recap, the definition of $O(g(n)$, called \lq{}big oh\rq{} of
$g(n)$, is
\begin{equation}
O(g(n))=\{f(n)| \exists n_0>0\in {\bf N}\mbox{ and }c>0\in {\bf R}\mbox{ with }|f(n)|\le c|g(n)|\,\forall n\ge n_0\}
\end{equation}
or, equivalently
\begin{equation}
f(n)\in O(g(n))\iff \lim_{n\rightarrow \infty}\frac{f(n)}{g(n)}<\infty
\end{equation}

In practice, if
\begin{equation}
T(n)=a_rn^r+a_{r-1}n^{r-1}+\ldots+a_1n+a_0
\end{equation}
then $T(n)\in O(n^r)$. 

As we saw, the logarithm has the funny property that it goes to infinity, but it
does so slower than $n$:
\begin{equation}
\lim_{n\rightarrow \infty}\frac{\log_2{n}}{n}=0
\end{equation}
Now, just as $\log_2{n}$ grows very slowly, $2^n$
grows very fast,
\begin{equation}
\lim_{n\rightarrow \infty}\frac{n^r}{2^n}=0
\end{equation}
for any finite value of $r$, worse still is $n!$, pronounced $n$-factorial
\begin{equation}
n!=n(n-1)(n-2) . . . 1
\end{equation}
If your algorithm is in $O(n!)$ you will probably need a different
algorithm. A table of different values is given as
Table~\ref{table_n_values}, mostly to emphasis how quickly $n!$ gets
big.

Now, in mathematics we call something \lq{}an abuse of notation\rq{}
if it is common to write something that doesn't quite make sense but
acts as a shorthand for something that does. Now $O(g(n))$ is a set of
functions whose large $n$ behavior is bounded by $g(n)$ so in
algorithms, being in $O(g(n))$ is a property of $T(n)$, the formula
for the running time of the algorithm. However, it is a standard abuse
of notation to say an algorithm is $O(g(n))$ for some $g(n)$ if
$T(n)\in O(g(n))$ for all cases. This is another way of saying that
the worst behavior of the algorithm isn't any worse than the behavior
of $g(n)$ for large $n$ so all possible $T(n)$ are elements of
$O(g(n))$. Finding $O(g(n))$ for an algorithm will be refered to as
\lq{}finding the big-oh complexity\rq{}. In the context of a more precise approach to the topic we might talk about \lq{}finding the asymptotic complexity\rq{}.


\begin{figure}
% GNUPLOT: LaTeX picture with Postscript
\begingroup
  \makeatletter
  \providecommand\color[2][]{%
    \GenericError{(gnuplot) \space\space\space\@spaces}{%
      Package color not loaded in conjunction with
      terminal option `colourtext'%
    }{See the gnuplot documentation for explanation.%
    }{Either use 'blacktext' in gnuplot or load the package
      color.sty in LaTeX.}%
    \renewcommand\color[2][]{}%
  }%
  \providecommand\includegraphics[2][]{%
    \GenericError{(gnuplot) \space\space\space\@spaces}{%
      Package graphicx or graphics not loaded%
    }{See the gnuplot documentation for explanation.%
    }{The gnuplot epslatex terminal needs graphicx.sty or graphics.sty.}%
    \renewcommand\includegraphics[2][]{}%
  }%
  \providecommand\rotatebox[2]{#2}%
  \@ifundefined{ifGPcolor}{%
    \newif\ifGPcolor
    \GPcolorfalse
  }{}%
  \@ifundefined{ifGPblacktext}{%
    \newif\ifGPblacktext
    \GPblacktexttrue
  }{}%
  % define a \g@addto@macro without @ in the name:
  \let\gplgaddtomacro\g@addto@macro
  % define empty templates for all commands taking text:
  \gdef\gplbacktext{}%
  \gdef\gplfronttext{}%
  \makeatother
  \ifGPblacktext
    % no textcolor at all
    \def\colorrgb#1{}%
    \def\colorgray#1{}%
  \else
    % gray or color?
    \ifGPcolor
      \def\colorrgb#1{\color[rgb]{#1}}%
      \def\colorgray#1{\color[gray]{#1}}%
      \expandafter\def\csname LTw\endcsname{\color{white}}%
      \expandafter\def\csname LTb\endcsname{\color{black}}%
      \expandafter\def\csname LTa\endcsname{\color{black}}%
      \expandafter\def\csname LT0\endcsname{\color[rgb]{1,0,0}}%
      \expandafter\def\csname LT1\endcsname{\color[rgb]{0,1,0}}%
      \expandafter\def\csname LT2\endcsname{\color[rgb]{0,0,1}}%
      \expandafter\def\csname LT3\endcsname{\color[rgb]{1,0,1}}%
      \expandafter\def\csname LT4\endcsname{\color[rgb]{0,1,1}}%
      \expandafter\def\csname LT5\endcsname{\color[rgb]{1,1,0}}%
      \expandafter\def\csname LT6\endcsname{\color[rgb]{0,0,0}}%
      \expandafter\def\csname LT7\endcsname{\color[rgb]{1,0.3,0}}%
      \expandafter\def\csname LT8\endcsname{\color[rgb]{0.5,0.5,0.5}}%
    \else
      % gray
      \def\colorrgb#1{\color{black}}%
      \def\colorgray#1{\color[gray]{#1}}%
      \expandafter\def\csname LTw\endcsname{\color{white}}%
      \expandafter\def\csname LTb\endcsname{\color{black}}%
      \expandafter\def\csname LTa\endcsname{\color{black}}%
      \expandafter\def\csname LT0\endcsname{\color{black}}%
      \expandafter\def\csname LT1\endcsname{\color{black}}%
      \expandafter\def\csname LT2\endcsname{\color{black}}%
      \expandafter\def\csname LT3\endcsname{\color{black}}%
      \expandafter\def\csname LT4\endcsname{\color{black}}%
      \expandafter\def\csname LT5\endcsname{\color{black}}%
      \expandafter\def\csname LT6\endcsname{\color{black}}%
      \expandafter\def\csname LT7\endcsname{\color{black}}%
      \expandafter\def\csname LT8\endcsname{\color{black}}%
    \fi
  \fi
  \setlength{\unitlength}{0.0500bp}%
  \begin{picture}(7200.00,5040.00)%
    \gplgaddtomacro\gplbacktext{%
      \csname LTb\endcsname%
      \put(594,440){\makebox(0,0)[r]{\strut{} 0}}%
      \put(594,1059){\makebox(0,0)[r]{\strut{} 10}}%
      \put(594,1679){\makebox(0,0)[r]{\strut{} 20}}%
      \put(594,2298){\makebox(0,0)[r]{\strut{} 30}}%
      \put(594,2917){\makebox(0,0)[r]{\strut{} 40}}%
      \put(594,3536){\makebox(0,0)[r]{\strut{} 50}}%
      \put(594,4156){\makebox(0,0)[r]{\strut{} 60}}%
      \put(594,4775){\makebox(0,0)[r]{\strut{} 70}}%
      \put(726,220){\makebox(0,0){\strut{} 0}}%
      \put(1739,220){\makebox(0,0){\strut{} 1}}%
      \put(2752,220){\makebox(0,0){\strut{} 2}}%
      \put(3765,220){\makebox(0,0){\strut{} 3}}%
      \put(4777,220){\makebox(0,0){\strut{} 4}}%
      \put(5790,220){\makebox(0,0){\strut{} 5}}%
      \put(6803,220){\makebox(0,0){\strut{} 6}}%
    }%
    \gplgaddtomacro\gplfronttext{%
      \csname LTb\endcsname%
      \put(5816,4602){\makebox(0,0)[r]{\strut{}$2^x$}}%
      \csname LTb\endcsname%
      \put(5816,4382){\makebox(0,0)[r]{\strut{}$x^2+1$}}%
    }%
    \gplbacktext
    \put(0,0){\includegraphics{exp.eps}}%
    \gplfronttext
  \end{picture}%
\endgroup

\caption{This shows $2^x$ and $x^2-1$ plots for $x\in[0,6]$, clearly $2^x$ quickly overtakes $x^2+1$, this will happen for any power of $x$. \label{fig_exp}}
\end{figure}

\begin{table}
\begin{tabular}{l|cccccc}
        $n$    &1   &2&4   &16  &128&1024\\
$\log{n}$      &0   &1&2   &4   &7  &10\\
$n\log{n}$     &0   &2&8   &64  &896&10240\\
$n^2$     &1   &4&16&256&16384&1048576\\
$2^n$     &2   &4&16&65536&$3.4\times 10^{38}$&$1.8\times 10^{307}$\\
$n!$      &1   &2&24&$2.1\times 10^{13}$&$3.85\times 10^{305}$&$5.4\times10^{2369}$
\end{tabular}
\vskip 1cm The website\\ {\tt
  http://markknowsnothing.com/cgi-bin/calculator.php}\\ was used for
the $2^n$ calculations and\\ {\tt
  http://www.calculatorsoup.com/calculators/discretemathematics/factorials.php}\\ for
the $n!$ calculations; these give answers even when the answer is very
large. Another easy way to calculate with large numbers is to use Python.

\caption{Different values of $n$ for some functions.  \label{table_n_values}
}
\end{table}

\subsubsection*{Other big Letter notations, small oh notation.}

There is another set, $\Omega(g(n))$ with a definition similar to
$O(g(n))$ that is used for describing the best case behavior. This
requires a lower bound rather than an upper bound, so the obvious
definition is
\begin{equation}
\Omega(g(n))=\{f(n)| \exists n_0>0\in {\bf N}\mbox{ and }c>0\in {\bf R}\mbox{ with }|f(n)|\ge c|g(n)|\,\forall n\ge n_0\}
\end{equation}
or
\begin{equation}
f(n)\in \Omega(g(n))\iff \lim_{n\rightarrow \infty}\frac{f(n)}{g(n)}>0
\end{equation}
in other words, the same thing, but with the $\le$ symbol replaced by
a $\ge$. In fact, there is some ambiguity about this definition,
number theorists use a slightly different one. Either way, it isn't
used very often in computer science because algorithms are very
frequently $\Omega(1)$; in the best case scenario the problem is in
some sense already solved, the array already sorted for example, and
the algorithm finishes in one step. 

There is also a set of functions that are both bounded above and below
by the same $g(n)$
\begin{equation}
\Theta(g(n))=\Omega(g(n))\cap O(g(n))
\end{equation}
This works because it is possible for
\begin{equation}
c_1 g(n)\le f(n)\le c_2g(n)
\end{equation}
for different $c_1$ and $c_2$. It would be very unusual for this to
apply to an algorithm, it would mean that $T(n)$ has the same behavior
for large $n$ no matter whether it is the best case or the worst case
scenario. There is a na\"ive largest element function in
Table~\ref{c_largest_linear_search} which is $\Theta(n)$. It searches
for the largest value in an unsorted array by looking at each element
in turn. In fact, for a completely unsorted array this is the best
algorithm, but, in practice, if finding the largest element in a set
is an important and frequent procedure, a special data structure,
called a heap, is used to keep track of which element is largest.

\begin{table}
\begin{lstlisting}[numbers=left]
int search(int a[],int n)
{
   int i;
   int best_val=a[0];

   for(i=1;i<n;i++){
      if(a[i]>best_val)
         best_val= a[i];
   }

  return best_val;
}
\end{lstlisting}
\caption{Search for the largest element in an unsorted list. This
  function searches all the elements to see which is the largest, the
  inner loop always runs $n-1$ times since it doesn't know until it
  has looked at every element which is going to be the largest. This
  program is implemented as {\tt
    find\_largest.c}.\label{c_largest_linear_search}.}
\end{table}

Finally, little oh notation is a stricter version of big Oh notation
that is used in some more mathematical context, basically $f(n)\in
o(g(n))$ is $f(n)$ is greater than $cg(n)$ for any choice of $c$, if $n$
is large enough:
\begin{equation}
o(g(n))=\{f(n)| \exists n_0>0\in {\bf N}\mbox{ so that }|f(n)|>
c|g(n)|\,\forall n\ge n_0 \mbox{ and }\forall\,c>0\in {\bf R}\}
\end{equation}


\end{document}
